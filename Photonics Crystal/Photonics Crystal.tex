\documentclass[%
 reprint,
 amsmath,amssymb,
 aps,
rmp,
]{revtex4-1}

\usepackage{graphicx}% Include figure files
\usepackage{dcolumn}% Align table columns on decimal point
\usepackage{bm}% bold math

\begin{document}

\preprint{APS/123-QED}

\title{Photonics Crystal}% Force line breaks with \\

\author{Yang Long}
 \email{longyang\_123@yeah.net}
\affiliation{
School of Physics Sciences and Engineering, Tongji University, Shanghai 200092, China
}

\date{\today}

\begin{abstract}
A basic review of Photonics Crystal
\end{abstract}

\maketitle

\section{Introduction}

\section{Band Structure}

\section{Tight Binding Treatments}
For example, we will consider the scalar case of a 2D periodic array of $N$ infinitely long dielectric cylinders in vacuum, with periodic boundary conditions and with the incident plane wave $\bm{E}$ polarized. We assume the normalized electric field for each band to be given by
\begin{equation}
E_n(\bm{r},\bm{k}) = \frac{c_n^1(k)}{\sqrt{N}}e^{i\bm{k}\bm{r}} + \frac{c_n^2(k)}{\sqrt{N}} \sum_{\bm{R}} \Psi_n(\bm{r}-\bm{R})e^{i\bm{k}\bm{R}}
\end{equation}
where $n=0,1,2...$ is the band's index and $\Psi_n(\bm{r}-\bm{R})$ with an angular symmetry $\Psi \sim cos(n\theta)$ stands for the wave function of the $n$th resonance localized at $\bm{R}$.

\subsection{Linear combination of atomic orbitals(LCAO)}
\begin{equation}
H = \begin{pmatrix}
\langle s|H|s \rangle & \langle s|H|p_x \rangle & \langle s|H|p_y \rangle & \langle s|H|p_z \rangle \\
\langle p_x|H|s \rangle & \langle p_x|H|p_x \rangle & \langle p_x|H|p_y \rangle & \langle p_x|H|p_z \rangle \\
\langle p_y|H|s \rangle & \langle p_y|H|p_x \rangle & \langle p_y|H|p_y \rangle & \langle p_y|H|p_z \rangle  \\
\langle p_z|H|s \rangle & \langle p_z|H|p_x \rangle & \langle p_z|H|p_y \rangle & \langle p_z|H|p_z \rangle 
\end{pmatrix}
\end{equation}

\begin{equation}
h_{\lambda}(r) = \begin{pmatrix}
\langle s|H|s \rangle & \lambda = ss\sigma \\
\langle s|H|p_\parallel \rangle & \lambda = sp\sigma \\
\langle p_\parallel|H|p_\parallel \rangle & \lambda = pp\sigma \\
\langle p_{\perp}|H|p_{\perp} \rangle & \lambda = pp\pi \\
\end{pmatrix}
\end{equation}

\section{Applications}

\end{document}
%
% ****** End of file apssamp.tex ******