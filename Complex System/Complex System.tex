\documentclass[%
 reprint,
 amsmath,amssymb,
 aps,
rmp,
]{revtex4-1}

\usepackage{graphicx}% Include figure files
\usepackage{dcolumn}% Align table columns on decimal point
\usepackage{bm}% bold math

\begin{document}

\preprint{APS/123-QED}

\title{Complex System}% Force line breaks with \\

\author{Yang Long}
 \email{longyang\_123@yeah.net}
\affiliation{
School of Physics Sciences and Engineering, Tongji University, Shanghai 200092, China
}

\date{\today}

\begin{abstract}
A basic review of Complex System
\end{abstract}

\maketitle

\section{Introduction}

\section{Definition}

\section{Dynamic Equation}
Now, consider a dynamical network consisting of $N$ identical linearly and diffusively coupled nodes, with each node being an $n$-dimensional dynamical system. The state equations of the network are:
\begin{equation}
\dot{x}_i = f(x_i) + c \sum_{j=1}^{N} a_{ij} \Gamma x_j,\qquad i = 1,2,...,N.
\label{eq:lineardynamicequation}
\end{equation}

\section{Properties}

\section{Numerical calculation and convergence}

\subsection{Runge Kutta}

\subsection{Finite difference time domain}

\section{Classical Models}
\subsection{Discrete nonlinear Schr$\ddot{o}$dinger equation}
\begin{equation}
i \frac{d\Psi_n}{dt} = \sum_m V_{n,m} \Psi_m - \gamma |\Psi_n|^2 \Psi_n
\end{equation}

\begin{equation}
i \frac{d\Psi}{d\tau} = M \Psi - \chi D(|\Psi|^2) \Psi
\end{equation}

\begin{equation}
M \Psi = \lambda \Psi
\end{equation}

\section{Applications}

\end{document}
%
% ****** End of file apssamp.tex ******