\documentclass[%
 reprint,
 amsmath,amssymb,
 aps,
rmp,
]{revtex4-1}

\usepackage{graphicx}% Include figure files
\usepackage{dcolumn}% Align table columns on decimal point
\usepackage{bm}% bold math

\begin{document}

\preprint{APS/123-QED}

\title{Multiphysics}% Force line breaks with \\

\author{Yang Long}
 \email{longyang_123@yeah.net}
\affiliation{
School of Physics Sciences and Engineering, Tongji University, Shanghai 200092, China
}

\date{\today}

\begin{abstract}
A basic review of Multi-physics simulation
\end{abstract}

\maketitle

\section{Introduction}

\section{Universal Mathematical Models}

\subsection{Partial Differential Equation(PDE)}
Coefficient Form:
\begin{equation}
e\frac{\partial^2 u}{\partial t^2} + d \frac{\partial u}{\partial t} + \nabla\cdot(-c \nabla u-\alpha u +\gamma) + \beta\cdot \nabla u + \alpha u = f
\label{eq:coefficientform}
\end{equation}

General Form:
\begin{equation}
e\frac{\partial^2 u}{\partial t^2} + d \frac{\partial u}{\partial t} + \nabla\cdot\Gamma = f
\label{eq:generalform}
\end{equation}

Weak Form:
\begin{equation}
0 = \int_{\Omega} g \partial V
\label{eq:weakform}
\end{equation}

\subsection{Boundary}

\subsection{Source}

\subsection{Classical PDEs}
Wave equation:
\begin{equation}
e\frac{\partial^2 u}{\partial t^2} + \nabla\cdot(-c\nabla u) =f
\label{eq:waveequation}
\end{equation}

Laplace Equation:
\begin{equation}
\nabla\cdot(-\nabla u)=0
\label{eq:laplaceequation}
\end{equation}

Poisson Equation:
\begin{equation}
\nabla\cdot(- c\nabla u)=f
\label{eq:poissonequation}
\end{equation}

Helmholtz Equation:
\begin{equation}
\nabla\cdot(- c\nabla u) + \alpha u =f
\label{eq:helmholtzequation}
\end{equation}

Heat Equation:
\begin{equation}
d \frac{\partial u}{\partial t} + \nabla\cdot(- c\nabla u) =f
\label{eq:heatequation}
\end{equation}

Convection-Diffusion Equation:
\begin{equation}
d \frac{\partial u}{\partial t} + \nabla\cdot(- c\nabla u)  + \beta\cdot\nabla u=f
\label{eq:convectiondiffusionequation}
\end{equation}

\section{Physical Field}

\subsection{The motion of fluids}
The Navier-Stokes equations govern the motion of fluids and can be seen as Newton's second law of motion for fluids.
\begin{equation}
\rho (\frac{\partial \bm{u}}{\partial t} + \bm{u}\cdot\nabla\bm{u})=  -\nabla\rho + \nabla\cdot(\mu(\nabla\bm{u} + (\nabla\bm{u})^T) - \frac{2}{3}\mu(\nabla\cdot\bm{u})\bm{I}) + \bm{F}
\label{eq:navierstokesequation}
\end{equation}
where $\bm{u}$ is the fluid velocity, $p$ is the fluid pressure, $\rho$ is the fluid density, and $\mu$ is the fluid dynamic viscosity. There are four terms in Navier-stokes equations: (1) $\rho (\frac{\partial \bm{u}}{\partial t} + \bm{u}\cdot\nabla\bm{u})$: the inertial forces; (2)$-\nabla\rho$: the pressure forces; (3)$\nabla\cdot(\mu(\nabla\bm{u} + (\nabla\bm{u})^T) - \frac{2}{3}\mu(\nabla\cdot\bm{u})\bm{I})$: viscous forces; (4)$\bm{F}$: the external forces applied to the fluid.

Solving together with the continuity equation, we can get:
\begin{equation}
\frac{\partial \rho}{\partial t} + \nabla\cdot(\rho\bm{u})=0
\label{eq:continueNavierStokesequation}
\end{equation}
The Navier-Stokes equations represent the conservation of momentum, while the continuity equation represents the conservation of mass.

\end{document}
%
% ****** End of file apssamp.tex ******